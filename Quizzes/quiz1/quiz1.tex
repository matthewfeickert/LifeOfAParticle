
\documentclass[12pt]{article}
%	options include 12pt or 11pt or 10pt
%	classes include article, report, book, letter, thesis

\usepackage[margin=0.5in]{geometry}

\setlength{\parindent}{0pt}

\usepackage{hyperref}

%for writing of code in blocks like
%\begin{lstlisting}
%   .......
%\end{lstlisting}
\usepackage{listings}
\usepackage{color}
\usepackage{enumitem}

\definecolor{dkgreen}{rgb}{0,0.6,0}
\definecolor{gray}{rgb}{0.5,0.5,0.5}
\definecolor{mauve}{rgb}{0.58,0,0.82}

\lstset{frame=tb,
  language=C++,
  aboveskip=3mm,
  belowskip=3mm,
  showstringspaces=false,
  columns=flexible,
  basicstyle={\small\ttfamily},
  numbers=none,
  numberstyle=\tiny\color{gray},
  keywordstyle=\color{blue},
  commentstyle=\color{dkgreen},
  stringstyle=\color{mauve},
  breaklines=true,
  breakatwhitespace=true,
  tabsize=3
}
%%%%%%%%%%%%%%%%%%%%%%

\title{Life of a Particle : Quiz 1}
\author{Sam Meehan}
\date{Due Date : 5 January 2017}

\begin{document}
\maketitle


\textbf{Guidelines} 
\newline
Choose one of the two tasks below.
\newline
\newline
For many common tasks, there are built in functions already existing in python, or which can be imported from the \small{\ttfamily{math}} or \small{\ttfamily{numpy}} modules.  In particular, the ones you may have used before are \small{\ttfamily{min(LIST)}}, \small{\ttfamily{max(LIST)}}, \small{\ttfamily{sum(LIST)}}, \small{\ttfamily{len(LIST)}}.  However, these are not necessary for ``good'' programmers.  Nearly all programs can be written using the following small set of operations 
\begin{itemize}[noitemsep]
\item $+$ (add)
\item $*$ (multiply
\item $/$ (divide)
\item $=$ (assignment)
\item $==$ (equals comparison)
\item $<$ (less than comparison)
\item $>$ (greater than comparison)
\end{itemize}
In addition to these operators, it is taken for granted in programming that you can use the following features as well
\begin{itemize}[noitemsep]
\item variables and lists - for storing the initial dataset
\item for loops
\item if statements
\item print statements - for viewing your code
\end{itemize}
In this quiz, these are the only things that may appear.   If you determine that you absolutely need some other function or operator, then include a comment clearly describing why this is the case.
\newline
Finally, you are not allowed to ``hard code'' in your program, meaning that there cannot be code that you must manually change each time you run it.  An example of this is the length of a list.  If I have a list \textit{[1,4,2,5,3,6]}, and you want to use the number of items in the list in your code, then the number ``6'' may not appear in your code.  
\newline
\newline
\textbf{Task 1}
\newline
If I give you an integer \textit{Q}, write a single program that finds the average of all of the integers between \textit{0} and that integer \textit{Q}, including \textit{Q} itself.  Be sure that your code works for both positive and negative values of \textit{Q}.
\newline
\newline
\textbf{Task 2}
\newline
Given this set of data (you can copy and paste it into an array if you like)
\newline
\newline
[71, 51, 32, 62, 84, 109, 43, 92, 72, 41, 102, 80, 72, 69, 46, 94, 52, 95, 90, 72, 63, 70, 34, 80, 78, 34, 31, 37, 26, 41, 42, 107, 33, 108, 108, 75, 66, 23, 90, 53, 24, 70, 26, 41, 93, 24, 71, 39, 48, 66, 97, 107, 77, 71, 67, 39, 38, 107, 96, 92, 84, 46, 60, 95, 87, 90, 92, 63, 78, 78, 84, 107, 70, 108, 32, 36, 93, 108, 49, 72, 56, 43, 30, 56, 51, 97, 45, 92, 40, 43, 49, 83, 98, 28, 99, 97, 102, 89, 58, 87]
\newline
\newline
write a program in python which computes the (i) \textit{maximum value} and the (ii) \textit{minimum value}.













\end{document}